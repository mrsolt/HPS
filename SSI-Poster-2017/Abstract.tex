\documentclass[11pt]{amsart}
\usepackage{geometry}                % See geometry.pdf to learn the layout options. There are lots.
\geometry{letterpaper}                   % ... or a4paper or a5paper or ... 
%\geometry{landscape}                % Activate for for rotated page geometry
%\usepackage[parfill]{parskip}    % Activate to begin paragraphs with an empty line rather than an indent
\usepackage{graphicx}
\usepackage{amssymb}
\usepackage{epstopdf}
\DeclareGraphicsRule{.tif}{png}{.png}{`convert #1 `dirname #1`/`basename #1 .tif`.png}

\title{Abstract for Poster:  Heavy Photon Search}
\author{Sebouh Paul, Miriam Diamond}
%\institute{College of William and Mary, SLAC National Accelerator Laboratory}
%\conference{SLAC Summer Institute 2017}
%\email{sebouh.paul@gmail.com, mdiamond@slac.stanford.edu}
%\date{}                                           % Activate to display a given date or no date

\begin{document}
\maketitle
%\section{}
%\subsection{}
Compelling motivations, including promising classes of dark matter models compatible with various observed astrophysical anomalies, motivate the existence of a massive gauge boson carrier of a $U'(1)$ gauge symmetry beyond the Standard Model. This hypothesized ``heavy photon" (abbreviated $A'$) would interact feebly with the SM, through kinetic mixing with the SM photon.  Our experiment seeks to produce the $A'$ in a laboratory setting by striking a tungsten foil target with a continuous electron beam.  An $A'$ may be produced in a process analogous to bremsstrahlung, and then decay into an $e^+e^-$ pair.   Our detector setup consists of a silicon vertex tracker surrounded by a large magnet for tracking the pair, and an electronic calorimeter for triggering.  To find the $A'$ signal amongst the much larger QED background, we use resonance search (bump-hunt) and displaced-vertex search techniques.  We have taken data with two beam energies: 1.05 GeV during our 2015 engineering run, and 2.3 GeV in our 2016 production run. This poster describes the experimental setup and presents preliminary results for the two datasets, as well as projected reach for future running.  


\end{document}  
