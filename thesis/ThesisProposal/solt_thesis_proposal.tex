\documentclass[12pt]{article}
\usepackage[top=1.00in, bottom=1.00in, left=1.00in, right=1.00in]{geometry}

\begin{document}
	\begin{center}
		\textbf{The Search for Hidden Sectors in the HPS Experiment} \\
		\emph{Matt Solt}				            \\
		Stanford University    	\\
	\end{center}

    \textbf{The Heavy Photon Search Experiment} \\
    \\
    	The Heavy Photon Search (HPS) experiment at Jefferson Lab is searching 
    for a new $U(1)$ vector boson (``heavy photon'', ``dark photon'' or $A'$)
    in the mass range of 20-500 MeV/c$^{2}$. An $A'$ in this mass range is
    theoretically favorable and may also mediate dark matter interactions.  
    The $A'$ couples to the ordinary photon through kinetic mixing, which 
    induces its coupling to electric charge. Since heavy photons couple to
    electrons, they can be produced through a process analogous to 
    bremsstrahlung, subsequently decaying to an $e^{+}e^{-}$, which can be
    observed as a narrow resonance above the dominant QED trident background.
    For suitably small couplings, heavy photons travel detectable distances
    before decaying, providing a second signature. Using the CEBAF electron
    beam at Jefferson Lab incident on a thin tungsten target, along with a 
    compact, large acceptance forward spectrometer consisting of a silicon
    vertex tracker and lead tungstate electromagnetic calorimeter, HPS is 
    accessing unexplored regions in the mass-coupling phase space. \\
    \\

    \textbf{Thesis Goals} \\
    \\
    	My thesis will consist of three major parts - operations, service, and data analysis. \\
    \\
    1. \textbf{Operations} - HPS requires students to participate fully in running the experiment including calibrations, data quality, and preparing data for analysis. I was heavily involved in operations, calibrations, data quality and preparation for the second HPS engineering run which took place in the spring of 2016 using a 2.3 GeV beam with 200 nA of current. I will also be very involved in the planned 2019 running, pending approval of running.\\
    \\
    2. \textbf{Service} - My thesis work will consist of participating in minor upgrades. Specifically, HPS is planning an upgrade by adding an additional silicon tracking layer (Layer 0) between the target and current first layer. I have already conducted a detailed simulation and have shown that Layer 0 improves our vertex resolution, and hence dramatically improves our reach. In the near future, I will contribute to the construction and testing of Layer 0 upgrade as well as its commissioning during the 2019 run. \\
    	I am also involved in estimating HPS's potential for studying models other than the minimal $A'$ model for the current datasets and future runs. One such model is the strongly interacting massive particles (SIMPs). SIMPs would have a displaced vertex signal similar to an A', but would also have a missing energy signature due to the production of a dark pion which would not be detected.\\
    \\
    3. \textbf{Data Analysis} - My data analysis will consist of the vertex search for the 2016 dataset. The vertex search involves looking at displaced vertices on the order of 1-10 cm, which occur for heavy photons with sufficiently small couplings. The current vertexing analysis for the 2015 dataset consisted of a basic cut and count method; however, there is room for much improvement for this analysis. First, this analysis only works for requiring first layer hits for both the electron and positron. For longer lived $A'$s, it is necessary to look at events that do not meet this requirement, which opens up the possibility for more complicated backgrounds that need to be understood. In addition, instead of the simple cut and count method, I will explore an ``event by event" analysis in which I will try to exploit event-by-event mass resolution, vertex resolution, radiative fraction, and $x=E_{e^+ e^-}/E_{beam}$.
    \\
    \\
    \\
%\begin{verbatim}
%________________________________________________________________
%\end{verbatim}
%    Matthew R. Solt, PhD Candidate
%    \\
%    \\
%\begin{verbatim}
%________________________________________________________________
%\end{verbatim}
%    Philip Schuster, Adviser


\end{document}
