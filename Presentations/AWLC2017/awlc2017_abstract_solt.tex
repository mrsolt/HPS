\documentclass[12pt]{article}
\usepackage[top=1.00in, bottom=1.00in, left=1.00in, right=1.00in]{geometry}

\begin{document}
	\begin{center}
		\textbf{Heavy Photon Search} \\
		\emph{Matt Solt}				            \\
		SLAC National Accelerator Laboratory    	\\
	\end{center}


    The Heavy Photon Search (HPS) experiment at Jefferson Lab is searching 
    for a new $U(1)$ vector boson (``heavy photon'', ``dark photon'' or $A'$)
    in the mass range of 20-500 MeV/c$^{2}$. An $A'$ in this mass range is
    theoretically favorable and may also mediate dark matter interactions.  
    The $A'$ couples to the ordinary photon through kinetic mixing, which 
    induces their coupling to electric charge. Since heavy photons couple to
    electrons, they can be produced through a process analogous to 
    bremsstrahlung, subsequently decaying to an $e^{+}e^{-}$, which can be
    observed as a narrow resonance above the dominant QED trident background.
    For suitably small couplings, heavy photons travel detectable distances
    before decaying, providing a second signature. Using the CEBAF electron
    beam at Jefferson Lab incident on a thin tungsten target, along with a 
    compact, large acceptance forward spectrometer consisting of a silicon
    vertex tracker and lead tungstate electromagnetic calorimeter, HPS is 
    accessing unexplored regions in the mass-coupling phase space. 

    The HPS engineering runs took place in spring of 2015 using a 1.056 GeV beam
    and in winter of 2016 using a 2.3 GeV beam. This talk will present an overview of the HPS 
    experiment and discuss the latest results.

\end{document}
